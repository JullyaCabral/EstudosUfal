\documentclass[a4paper,12pt]{article}
\usepackage[utf8]{inputenc}
\usepackage{amsmath, amssymb}
\usepackage{geometry}
\usepackage{lmodern}
\usepackage{graphicx}

\geometry{margin=2.5cm}

\title{Distribuições de Probabilidade}
\author{}
\date{}

\begin{document}

\maketitle

\section{Variáveis Aleatórias Discretas}

\subsection{Distribuição Uniforme Discreta}

\textbf{Conceito:} Cada resultado possui a mesma probabilidade de ocorrer. Se temos um conjunto de \( n \) valores, a chance de cada um aparecer é igual.

\textbf{Fórmula:}
\begin{equation}
P(X = x) = \frac{1}{n}
\end{equation}

\textbf{Exemplo Simples – Dado:}
\begin{itemize}
    \item Imagine um dado justo com 6 faces.
    \item Cada face tem a mesma chance:
    \begin{equation}
    P(X = x) = \frac{1}{6} \approx 0{,}1667, \quad x \in \{1,2,3,4,5,6\}
    \end{equation}
\end{itemize}

\textbf{Exemplo Complexo – Soma de Dois Dados:}
\begin{itemize}
    \item Há \( 6 \times 6 = 36 \) resultados possíveis, mas nem todas as somas têm a mesma frequência.
    \item Para a soma 7:
    \[
    (1,6), (2,5), (3,4), (4,3), (5,2), (6,1) \Rightarrow 6 \text{ combinações}
    \]
    \item A probabilidade de obter 7:
    \begin{equation}
    P(X = 7) = \frac{6}{36} = \frac{1}{6} \approx 0{,}1667
    \end{equation}
\end{itemize}

\textit{Observação:} Embora resultados individuais de um dado sejam uniformes, a soma de dois dados não é. Isso ilustra a diferença entre “distribuição uniforme” e “função de probabilidade de um resultado composto”.

\subsection{Distribuição de Bernoulli}

\textbf{Conceito:} Modela experimentos com dois resultados possíveis: sucesso (1) e fracasso (0).

\textbf{Fórmula:}
\begin{equation}
P(X = x) = p^x (1 - p)^{1 - x}, \quad x \in \{0, 1\}
\end{equation}

Onde:
\begin{itemize}
    \item \( p \) é a probabilidade de sucesso;
    \item \( 1 - p \) é a probabilidade de fracasso.
\end{itemize}

\textbf{Exemplo Simples – Lançamento de Moeda:}
\begin{itemize}
    \item \( p = 0{,}5 \)
    \item \( P(X = 1) = 0{,}5 \)
    \item \( P(X = 0) = 0{,}5 \)
\end{itemize}

\textbf{Exemplo Complexo – Teste de Medicamento:}
\begin{itemize}
    \item \( p = 0{,}8 \)
    \item \( P(X = 1) = 0{,}8 \)
    \item \( P(X = 0) = 0{,}2 \)
\end{itemize}

\subsection{Distribuição Binomial}

\textbf{Conceito:} Representa o número de sucessos em \( n \) ensaios independentes com probabilidade de sucesso \( p \).

\textbf{Fórmula:}
\begin{equation}
P(X = k) = \binom{n}{k} p^k (1 - p)^{n - k}, \quad k = 0, 1, \dots, n
\end{equation}
\begin{equation}
\binom{n}{k} = \frac{n!}{k!(n - k)!}
\end{equation}

\textbf{Exemplo Simples – 3 Lançamentos de Moeda:}
\[
P(X = 2) = \binom{3}{2} (0{,}5)^2 (0{,}5)^1 = 3 \times 0{,}25 \times 0{,}5 = 0{,}375
\]

\textbf{Exemplo Complexo – Testes Clínicos:}
\[
P(X = 7) = \binom{10}{7} (0{,}7)^7 (0{,}3)^3
\]
\[
\binom{10}{7} = \frac{10!}{7!3!} = 120
\]
\[
P(X = 7) = 120 \times (0{,}7)^7 \times (0{,}3)^3
\]

\subsection{Distribuição de Poisson}

\textbf{Conceito:} Modela eventos raros em intervalos fixos, com média \( \lambda \).

\textbf{Fórmula:}
\begin{equation}
P(X = k) = \frac{\lambda^k e^{-\lambda}}{k!}, \quad k = 0, 1, 2, \dots
\end{equation}

\textbf{Exemplo Simples – Acidentes por Dia:}
\[
\lambda = 3, \quad P(X = 2) = \frac{3^2 e^{-3}}{2!} = \frac{9 e^{-3}}{2}
\]

\textbf{Exemplo Complexo – Ligações em Call Center:}
\[
\lambda = 15, \quad P(X = 12) = \frac{15^{12} e^{-15}}{12!}
\]

\section{Variáveis Aleatórias Contínuas}

\subsection{Distribuição Normal}

\textbf{Conceito:} Distribuição simétrica em torno da média \( \mu \), com desvio padrão \( \sigma \).

\textbf{Fórmula:}
\begin{equation}
f(x) = \frac{1}{\sigma \sqrt{2 \pi}} \exp \left( -\frac{1}{2} \left( \frac{x - \mu}{\sigma} \right)^2 \right)
\end{equation}

\textbf{Exemplo Simples – Normal Padrão:}
\begin{itemize}
    \item \( \mu = 0 \), \( \sigma = 1 \)
    \item \( P(-1 \leq X \leq 1) \approx 68{,}27\% \)
\end{itemize}

\textbf{Exemplo Complexo – Altura da População:}
\[
\mu = 170, \quad \sigma = 10
\]
\[
z = \frac{190 - 170}{10} = 2
\]
\[
P(X > 190) = 1 - P(X \leq 190) \approx 1 - 0{,}9772 = 0{,}0228
\]

\subsection{Distribuição Exponencial}

\textbf{Conceito:} Modela o tempo entre eventos contínuos com taxa \( \lambda \).

\textbf{Fórmula:}
\begin{equation}
f(x) = \lambda e^{-\lambda x}, \quad x \geq 0
\end{equation}

\textbf{Exemplo Simples – Atendimento:}
\[
\lambda = \frac{1}{5} = 0{,}2, \quad P(X \leq 3) = 1 - e^{-0{,}6}
\]

\textbf{Exemplo Complexo – Espera Longa:}
\[
P(X > 10) = e^{-0{,}2 \cdot 10} = e^{-2} \approx 0{,}1353
\]

\section*{Relacionando com Scripts no R}

\begin{itemize}
    \item \textbf{Uniforme:} \texttt{runif(n, min, max)} \\
    Ex: \texttt{runif(100, 1, 6)}
    
    \item \textbf{Bernoulli e Binomial:} \texttt{rbinom(n, size, prob)} \\
    Ex: \texttt{rbinom(100, 1, 0.5)} (Bernoulli), \texttt{rbinom(100, 3, 0.5)} (Binomial)

    \item \textbf{Poisson:} \texttt{rpois(n, lambda)} \\
    Ex: \texttt{rpois(100, lambda = 3)}

    \item \textbf{Normal:} \texttt{rnorm(n, mean, sd)} \\
    Ex: \texttt{rnorm(100, mean = 170, sd = 10)}

    \item \textbf{Exponencial:} \texttt{rexp(n, rate)} \\
    Ex: \texttt{rexp(100, rate = 0.2)}
\end{itemize}

\section*{Conclusão}

Em resumo, cada distribuição tem sua aplicação e fórmula associada:

\begin{itemize}
    \item \textbf{Uniforme Discreta:} \( P(X = x) = \frac{1}{n} \)
    \item \textbf{Bernoulli:} \( P(X = x) = p^x (1-p)^{1-x} \)
    \item \textbf{Binomial:} \( P(X = k) = \binom{n}{k} p^k (1 - p)^{n - k} \)
    \item \textbf{Poisson:} \( P(X = k) = \frac{\lambda^k e^{-\lambda}}{k!} \)
    \item \textbf{Normal:} \( f(x) = \frac{1}{\sigma \sqrt{2\pi}} e^{ -\frac{1}{2} \left( \frac{x - \mu}{\sigma} \right)^2 } \)
    \item \textbf{Exponencial:} \( f(x) = \lambda e^{-\lambda x} \)
\end{itemize}

\end{document}
